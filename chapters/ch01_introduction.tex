%!TEX root = ../report.tex

\begin{document}
    \chapter{项目介绍}

    \section{项目综述}
        \begin{spacing}{2.5}
我们的项目是实现一个软光栅渲染器。实现的基本功能:顶点变换、裁剪、线框画线
	附加功能:有键盘鼠标交互的相机旋转移动、扫描线填充、三维模型读取、纹理贴图、点光源光照、天空盒子     	
        \end{spacing}


    \subsection{...}

    \subsection{...}


    \section{开发环境}
	        \begin{spacing}{2.5}
由于我们的组员用的都是Mac,使用的C++开发环境为Xcode,具体版本为 Xcode Version 11.3 (11C29)
	        \end{spacing}


	
    \subsection{...}

    

    \subsection{...}




    \section{第三方依赖库}


    \subsection{SDL2}
\begin{spacing}{2.5}
   SDL2(Simple DirectMedia Layer 2)是一套开放源代码的跨平台多媒体开发库,使用C语言写成。其主要用于游戏开发中的多媒体处理,如视频渲染,音频播放,鼠标/键盘控制等操作。它对外接供了一套统一的接口,但在内部,它会根据不同平台调用不同的底层API库。
   在Mac平台下,调用的是OpenGL。在本项目中,我们调用了SDL2实现了窗口的创建、键盘敲击事件监听、鼠标按下与拖拽的监听等功能。
\end{spacing}



    \subsection{Assimp}
        \begin{spacing}{2.5}
        	Assimp库(Open Asset Import Library)是一个非常流行的模型导入库,可以导入复杂的三维模型,当使用Assimp导入模型的时候,它通常会将整个模型加载进一个场景(Scene),然后将场景载入为一系列的节点(Node),每个节点包含了场景对象中所储存数据的索引,每个节点都可以有任意数量的子节点。Assimp的具体结构会在后面的三维模型导入部分给出,在项目中我们利用了Assimp库进行同济大学校徽的三维模型导入,使其能加载进我们的场景中。
        \end{spacing}

    



    \subsection{Stb\_image}
            \begin{spacing}{2.5}
            	    stb\_image.h是Sean Barrett的一个非常流行的单头文件图像加载库,它能够加载大部分流行的图片文件格式,我们的项目利用了stb\_image进行贴图文件的读取和解析成纹理。
            \end{spacing}


    
\end{document}
