%!TEX root = ../report.tex

\begin{document}
    \chapter{附加功能}

    How you are planning to test/compare/evaluate your research.
    Criteria used.
	
	\section{键鼠交互的相机旋转移动}
	\begin{lstlisting}
		hello world!
	\end{lstlisting}
	\subsection{平移旋转变换}
	\begin{spacing}{2.5}
	其实,所谓相机的旋转移动可以看成相机固定不动,而物体在进行逆变换,例如,相机左移一个单位其实就是物体向右移动一个单位。因此要相机的旋转移动,只要实现物体对应的旋转移动即可。所用到的变换矩阵如下:
	\begin{enumerate}
		\item 平移变换矩阵
		\begin{equation}
		\begin{pmatrix}
			1 &  0&  0&0 \\ 
 			0&  1&  0& 0\\ 
 			0& 0 & 1 &0 \\ 
 			T_{x}&  T_{y}&  T_{z}& 1
		\end{pmatrix}
		\label{transfer_1}
		\end{equation}
		\item 	绕X轴旋转变换矩阵
		\begin{equation}
		\begin{pmatrix}
			 1&  0&  0 &0 \\ 
 			0&  cos\theta &  sin\theta& 0\\ 
 			 0& -sin\theta & cos\theta &0 \\ 
 			0&  0&  0& 1
		\end{pmatrix}
		\label{yrotate}
		\end{equation}
		\item 绕Y轴旋转变换矩阵
		\begin{equation}
		\begin{pmatrix}
			cos\theta &  0&  -sin\theta &0 \\ 
 			0&  1&  0& 0\\ 
 			sin\theta & 0 & cos\theta &0 \\ 
 			0&  0&  0& 1
		\end{pmatrix}
		\label{xrotate}	
		\end{equation}
	
		\item 绕Z轴旋转变换矩阵
		\begin{equation}
		\begin{pmatrix}
			cos\theta &  sin\theta &  0 &0 \\ 
 			-sin\theta &  cos\theta &  0& 0\\ 
 			0 & 0 & 1 &0 \\ 
 			0&  0&  0& 1
		\end{pmatrix}
		\label{zrotate}
		\end{equation}
		\item 缩放变换矩阵
		\begin{equation}
		\begin{pmatrix}
			S_{x}& 0 &  0 &0 \\ 
 			0 &  S_{y} &  0& 0\\ 
 			0 & 0 & S_{z} &0 \\ 
 			0&  0&  0& 1
		\end{pmatrix}
		\label{zrotate}
		\end{equation}
	\end{enumerate}
	\end{spacing}

    \section{区域填充}

    \section{纹理贴图}
    
    \section{光照}
    
    \section{三维模型}
    
    \section{天空盒子}
\end{document}
