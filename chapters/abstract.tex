%!TEX root = ../report.tex

\begin{document}
    \begin{abstract}
    \begin{spacing}{2.5}
    	\textbf{\textcolor{hbrs}{工作量描述}}\\
        \indent 这次的图形学项目是软光栅渲染器,我们从布置课题时就开始思考如何完成这一项目,一直到最终答辩,主要的理论研究和代码编写大概花了将近两个月的时间。
        前面一个月的时间主要是资料阅读与理论学习,我们阅读和学习的资料有:
        \begin{enumerate}
        	\item 《OpenGL渲染流水线之世界矩阵,相机变换矩阵,透视投影变换矩阵》\cite{14project}
        	\item 《用C\#实现一个简易的软件光栅化渲染器》\cite{axiyi}
        	\item 《从迷你光栅化渲染器的实现来看渲染流水线》\cite{16pipline}
        	\item 《实现逐顶点的ambient,diffuse and specular(ADS) shading phong 光照模型》 \cite{17light}
        	\item 《LearnOpenGL CN》\cite{20cn}
        	\item 《如何写一个软件渲染器》\cite{skywind}
        	\item 《OpenGL Transformation》\cite{18opengl}
        	\item 《SoftRender三部曲》\cite{15pipline}
        	\item 《如何开始用C++写一个光栅化渲染器》\cite{softZhihu}
        	\item 《透视投影(Perspective Projection)变换推导》\cite{13project}
        	\item 《DirectX视口变换矩阵详解》\cite{19DirectX}
        	\item 《OpenGL -- 渲染流水线之世界矩阵,相机变换矩阵,透视投影变换矩阵》\cite{1148525}
        	\item 《学习shader之前必须知道的东西之计算机图形学》\cite{12shader}
        \end{enumerate}
        后面一个月结合了所阅读的资料,主要进行代码的编写。
        后面一个月工作量如下:
        \begin{enumerate}
        	\item 找到并学习了一个vscode所编写的简单光栅渲染器A-Renderer\cite{zzzn}
        	\item 以学习和改编为目的,进行了代码移植。因为我们三位组员的电脑都是Mac,更习惯于使用Xcode作为C++开发的IDE,我们找到的A-Renderer是用VS开发的,其中平台运行环境、包的引入方式等都需要调整,实际操作中并不容易,花了很长一段时间才得以运行。
        \item 我们对代码架构进行了研读,并将代码与我们阅读的资料进行了结合加深对软光栅的理解。
        \item 理解代码和原理 后对代码架构进行了优化和改建。
        \item 之后,开始添加新功能:加入了三维模型的读取,引入第三方库assimp,学习了相关用法,使其可以读取obj格式的三维模型,并且用Maya制作了一个校徽的三维模型。
        \item 加入了纹理贴图功能。
        \item 加入了光照功能。
        \item 加入了键盘鼠标对物体坐标的移动效果,增加了项目的交互性,鼠标拖拽也使三维渲染更富有真实感。
        \item 撰写课程设计报告。
        \end{enumerate}
        本来我们打算还在软光栅的基础上与我们的报告项目《图形学在建筑领域的应用》做一个小的结合,添加一个功能:我们现有的3D模型还未实现贴图,而先有的贴图也是对简单正方体的贴图。我们打算制作一个较复杂的3D建筑模型,然后实现对这种复杂的三维建筑模型的贴图。但由于时间原因被搁置。这也是项目今后还能补充和完善的地方。\\
    \textbf{\textcolor{hbrs}{分工描述}}\\
    虽然三人是一直聚在一起学习和编程的,但三人也有各自的侧重:
    \begin{itemize}
    	\item 江晓湖:作为组长,参与并组织每次资料学习、编程、ppt制作、报告撰写。项目上侧重于顶点变换流水线、线框画线、相机变换、纹理贴图的研究。
    	\item 王毅诚:素材的制作与查找,参与报告撰写、ppt制作,侧重于项目的CVV裁剪、扫描填充、三维模型部分。
    	\item 王舸飞:参与ppt制作、报告撰写,侧重于项目的天空盒子、光照、纹理贴图部分。
    三人工作量基本一致。
    \end{itemize}
    \end{spacing}
    \end{abstract}
\end{document}
