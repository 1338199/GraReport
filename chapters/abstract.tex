%!TEX root = ../report.tex

\begin{document}
    \begin{abstract}
        这次的图形学项目是软光栅渲染器,我们从布置课题时就开始思考如何完成这一项目,一直到最终答辩,主要的理论研究和代码编写大概花了一个半月的时间。
        前面半个月的时间主要是参考的挑选与移植工作,后面一个月同时进行理论的学习和代码的编写。
        我们在这次项目中主要完成的工作量如下:\\
        1. 完成移植工作。因为我们三位组员的电脑都是Mac,更习惯于使用Xcode作为C++开发的IDE,我们找到的参考是用VS开发的,其中平台运行环境、包的引入方式等都需要调整,实际操作中并不容易,花了很长一段时间才得以运行。\\
        2. 理论的理解与推导。对于代码中出现的空间变换等空间理论,我们查阅了海量的资料进行反复推演,理解了其中的复杂变换,相关理论在本文档中都进行了说明\\
        3. 加入了三维模型的读取,引入第三方库assimp,学习了相关用法,使其可以读取obj格式的三维模型,并且用Maya制作了一个校徽的三维模型\\
        4. 加入了纹理填充功能,\\
        5. 加入了光照功能,\\
        6. 加入了键盘鼠标对物体坐标的移动效果,增加了项目的交互性,鼠标拖拽也使三维渲染更富有真实感\\
    \end{abstract}
\end{document}
